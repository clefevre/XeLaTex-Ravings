%! xelatex mal-urdu.tex
% To download: http://scripts.sil.org/cms/scripts/page.php?item_id=Scheherazade#1fd0063a
\documentclass[a4paper]{article}
\pagestyle{empty}
\usepackage{xltxtra}
\usepackage{fontspec}
\usepackage{polyglossia}
\setmainlanguage{english}
\setotherlanguage{urdu}
\newfontfamily\urdufont[Script=Arabic,Language=Urdu,Scale=1.5]{TLUrdu} % or Scheherazade after installing the font
\begin{document}
Text before. \texturdu{ښاغلی رشا صاحب وايي چي دالف-بې سېپاره مور ورته ویلې ده او پر لمانځه باندي یې هم درولئ دئ ؛ څه دېني کتابونه ماما او فارسي ورور ورته وويل ؛ بل ورور د خط او حساب دزدکړي دپاره پنسل او کتابچه پلاس کي ورکړه ، نور یې نو پخپله و خصوصي مطالعې ته ملا وتړل . ژوند یې اکثره دتجارت په مختلفو برخو کي تېر سوی دئ . د ژوند په نورو برخوکي له د زيات شمېر لکه د يات شمېر افغانانو غوندي ډیري سړې و تودې پر راغلي دي .} Text after.
\\
\texturdu{ د کندهار دښاروالۍ په انتخاباتو کي د دریمي ناحیې منتخب وکیل وټاکل سو، خو څنګه چي یې د ښاروال او نورو وکیلانو سره دخلګو د ګټو پر سر وخوړل او هم یې دښاروالۍ چاري د ښاریانو په ګټه نه بللې ، نو له وکالت څخه مستعفي سو. د جبههء پدر وطن په مو‌سسه کنګره کي کندهاریانو وکابل ته ولېږئ، هورې ددې لوی مجلس په غړو کي شامل و. په درو لویو جرګو کي یې دکندهار دخلګو دخوا استازيتوب کړئ دئ . همدارنګه یې دقبایلو په عالي جرګه کي دکندهار نمایندګي درلودل . دپارلمان په اولسي جرګه کي د کندهارد ښار دڅلورو ناحیو دخلګو له طرفه د رایو په اکثریت وکیل وټاکل سو او په اولسي جرګه کي د تجارت د کمیسیون غړی وټاکل سو .
 }\\

\texturdu{ په سنا کي یې دسنا د مجلس دمجلې د ژباړن په توګه هم دنده ترسره کړه . دسولي د جبهې د مو سسې کنګرې په غړو کي یې ګډون درلود ؛ ورپسې د سولي د جبهې دغړو له خوا د لمړي مرستيال په توګه وتاکل سو. په ولسي جرګه (پارلمان) کي د ټولو مخالفو ګروپو او اتحادیو له طرفه مشر او ریس وټاکل سو، دوی دا دمشرتوب دنده تر آخره وښاغلي ریشاء ته سپارلې وه.}
\\
\texturdu{ په ولسي جرګه کي به ریشاء تل په ټولنيزو مسایلو کي د ملت پله نیولې وه ، تل به یې د ملت د ګټوساتنه او دفاع کوله . دحکومت اوحکومت پلوه وکیلانو په وړاندي ولاړ و،او هغه وخت يې ويناوي په پرله پسې توګه د وخت په مطبوعاتو کي خپرېدلې .
}\\

\texturdu{ ( ښاغلی ریشاء حزبي نه دئ،په هیڅ ګونداو تنظیم کي غړیتوب نلري،دحق اوعدالت ملګری دئ ) }
\\
\texturdu{ ریشاء د تجارتي سفرونو په لړ کي و ایران، عراق، پاکستان، هنداو هانګ کانګ تللئ دئ او د جبههء پدر وطن په یوه رسمي سفر تشکند ، بخارا ، سمرقند ،مسکو او لیننګراډ ته هم تللئ دئ . }
\\
\texturdu{ ښاغلی ریشاء د ۱۳۲۶ ل راهیسي په شعر ویلو خوله او قلم پوري کړئ دئ . }
\\
\texturdu{ نوموړي دهغي لیونۍ میني له رویه ، چي خپل د هیواد، خلګو ، ژبي او فرهنګ سره یې لري، هیڅکله ځان ندئ سپمولئ او تل يې هڅه کړېده چي ددې خپلو خلکو او ټولني ته خپله معنوي پانګه ډالۍ کړي .}
\\
\texturdu{ښاغلی ريشاء د څو تېرو کلونو راپه دې خوا د پاکستان دکراچي په ښار کښي}\\
\texturdu{هستوګن وو . نوموړی د ۲۰۰۹ کال د سپټمبر د مياشتي پر ۲۰نېټه د ورپېښي}\\
\texturdu{ناروغۍ له مله په کراچي کې په حق ورسيېد ، روح دې ښاد وي .}\\

\end{document}
